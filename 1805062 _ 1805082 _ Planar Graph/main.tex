\documentclass{beamer}
\usepackage{graphicx}
\usetheme{Berlin}
\usepackage[utf8]{inputenc}
\usepackage{tikz}
\usepackage{multirow}
\usepackage{multicol}
\setbeamercovered{transparent}% Dim out "inactive" elements
\usepackage[export]{adjustbox}

\title{Planar Graph}

\author[1805062,1805082]{Ayesha Binte Mostofa \newline 1805062\newline Anup Bhowmik \newline 1805082}

\institute[BUET]
{
  Department of Computer Science and Engineering\\
  Bangladesh University of Engineering and Technology
}
\logo{\includegraphics[height=.8cm]{logo.png}}

\date{\today}
\begin{document}

\frame{\titlepage}

\begin{frame}
\frametitle{We are going to see}
\tableofcontents
\end{frame}

\begin{frame}{Motivation}
    \begin{itemize}
        \item PCB (Printed Circuit Board) is designed by planar graph. \newline
    \end{itemize}
    \begin{figure}
        \centering
        \includegraphics[height = 3cm]{PCB.jpeg}
        \caption{PCB Circuit}
    \end{figure}
\end{frame}

\tikzstyle{place}=[circle,draw=black!80,fill=black!30]

\section{Introduction}


\begin{frame}{Planar Graph}
\begin{block}{Definition}
A graph is called \textbf{{\color{blue}planar}} whether it can be drawn in the plane in such a way that no two edges cross.
\end{block}
\end{frame}

\begin{frame}{Planar Graph}
A clique on 4 nodes \newline
\begin{center}
\begin{tikzpicture}
    \node<1-> (s) at (0,-5) [place] {};
    \node<1-> (t) at (0,-7) [place] {};
    \node<1-> (r) at (3,-5) [place] {};
    \node<1-> (w) at (3,-7) [place] {};
    \draw (s)--(t) -- (w) -- (r) -- (s);
    \draw<1-1> (w)--(s);
    \draw<1-1> (t)--(r);
    \draw<2-4>[very thick,color = blue] (w)--(s);
    \draw<2-4>[very thick,color = blue] (t)--(r);
\end{tikzpicture}
\end{center}
\begin{itemize}
    \item<2-4> Let's see whether it's a planar graph or not
    \item<3-4> It has 6 edges, two of which crossing each other
    \item<4-4> So,it is not a Planar Graph \newline
\end{itemize}
\end{frame}


\begin{frame}{Planar Graph}
\begin{center}
\begin{tikzpicture}
    \node (s) at (0,-5) [place] {};
    \node (t) at (0,-7) [place] {};
    \node (r) at (3,-5) [place] {};
    \node (w) at (3,-7) [place] {};
    \draw (s)--(t) -- (w) -- (r) -- (s);
    \draw[very thick,color = blue] (s)--(w);
    \draw[very thick,color = blue] (t) .. controls (5,-9) .. (r);
\end{tikzpicture}
\end{center}
 Another clique on 4 nodes with six edges \newline
It's a planar Graph because no edges are crossing each other.
\newline
\end{frame}

\section{Faces of a planar graph}
\begin{frame}{Faces on a Planar Graph}
        \begin{block}{Definition}
       Consider a planar graph $G=(V,E)$ .A face is defined to be an area of the plane that is bounded by edges. A planar graph divides the planes into one or more \textbf{faces}. One of these faces always will be \emph{infinite}.
        \end{block}
\end{frame}

\begin{frame}{Finding Faces of a Planar Graph}
    \centering
    \begin{tikzpicture}
        \node [circle, minimum size = 10pt, very thick, fill = black!30] at (0,0) (s);
        \node [circle, minimum size = 10pt, very thick, fill = black!30] at (0,-3) (t);
        \node [circle, minimum size = 10pt, very thick, fill = black!30] at (-3,-6) (r);
        \node [circle, minimum size = 10pt, very thick, fill = black!30] at (3,-6) (v) ;
        
        \fill<1-1> [inner color = yellow!40, outer color = white, color = white] (0, -3.5) circle (3cm) ;
        \fill<1-1> [white] (s.center) -- (v.center) -- (r.center);
        
        \node<1-1> at (-6,-2) {\textbf{Infinite Face}}; 
        
        \node<2-4> at (-6,-2) {\textbf{Finite Face}}; 
        \fill<2-2> [top color = blue!30, bottom color = white] (s.center) -- (t.center) -- (r.center);

        \fill<3-3> [top color = red!30, bottom color = white] (s.center) -- (t.center) -- (v.center);
        \fill<4-4> [top color = green!30, bottom color = white] (t.center) -- (r.center) -- (v.center);
     
        \draw (s) -- (t);
        \draw (v) -- (t);
        \draw (r) -- (t);
        \draw (s) -- (r);
        \draw (v) -- (r);
        \draw (v) -- (s);

        \node [circle, minimum size = 10pt, very thick, fill = black!30] at (0,0) (s);
        \node [circle, minimum size = 10pt, very thick, fill = black!30] at (0,-3) (t);
        \node [circle, minimum size = 10pt, very thick, fill = black!30] at (-3,-6) (r);
        \node [circle, minimum size = 10pt, very thick, fill = black!30] at (3,-6) (v);
        
    \end{tikzpicture}
\end{frame}

\begin{section}{Euler’s Theorem}
    
    \begin{frame}{Euler’s Theorem on Planar Graph}
    \begin{itemize}
        \item Let G be a connected planar graph (drawn without crossing edges).
        \pause
        \item Define \newline $V = $ number of vertices
            \newline $E = $ number of edges
            \newline $F = $ number of faces, including the “infinite” face
        \pause
        \item Then $V - E + F = 2.$
    \end{itemize}
    
    \end{frame}
    
    \begin{frame}{Proof}
    \textbf{PROOF IDEA} \newline
    Proof by induction by the number of cycles \newline \newline
    \textbf{BASE CASE}
    \begin{itemize}
        \item G has no cycles \pause
        \item Since G is connected, it must be a tree. So, $e = v-1$ and $f = 1$.
        
        \begin{align*} 
v-e+f  &= v-(v-1)+1 \\ 
        &=  1 +1\\
        & = 2
\end{align*}
    \end{itemize}

    \end{frame}
    
    \begin{frame}{Proof -- Continued}
    \textbf{INDUCTION} 
    \begin{itemize}
        \item Let G has at least one cycle containing edge $e$ \pause
        \item let $G' = G - e$ \pause
    
        \begin{columns}
\column{0.6\textwidth}
\begin{center}
    \begin{tikzpicture}
        \node (a) at (-5,0) [place] {};
        \node (b) at (-4, .5) [place] {};
        \node (c) at (-2, .5) [place] {};
        \node (d) at (-1, 0) [place] {};
        \node (e) at (-2, -.5) [place] {};
        \node (f) at (-4, -.5) [place] {};
        
        \node <1-3> (o) at (0, 0)  {G};
        \node <4-> (o) at (0, 0)  {G'};
        
        \node <1-3>  at (-4.3, .1) {e};
        \draw <1-3> (a) -- (b) -- (c) -- (d) -- (e) -- (f) -- (a);
        
        \draw <4->  (b) -- (c) -- (d) -- (e) -- (f) -- (a);
    
    \end{tikzpicture}
    \end{center}

\column{0.4\textwidth}
 $v' = $ number of vertices
            \newline $e' = $ number of edges
            \newline $f' = $ number of faces
        
\end{columns}    \pause
    
        \newline
        \item Now, in $G', f' = f-1$ and $v' = v$ and $e' = e-1$ 
        \item \textbf{By induction hypothesis:} 
        $$v'-e' + f' = 2 $$
       $$ v-(e-1)+(f-1)=2 $$
       $$ v-e+f = 2 $$
        
    \end{itemize}
        
    \end{frame}
    
    \begin{frame}{Question}
    \begin{center}
    \textbf{Can you redraw this graph to alter the number of faces?} \newline
    
    \begin{tikzpicture}
        \node (a) at (-5,0) [place] {};
        \node (b) at (-4, 1) [place] {};
        \node (c) at (-2, 1) [place] {};
        \node (d) at (-1, 0) [place] {};
        \node (e) at (-2, -1) [place] {};
        \node (f) at (-4, -1) [place] {};
        \node (g) at (-3, 0) [place] {};


        \draw  (a) -- (b) -- (c) -- (d) -- (e) -- (f) -- (a);
        \draw (b) -- (g) -- (f);
     
    
    \end{tikzpicture}
    \end{center}
        
    \end{frame}
    
    \begin{frame}{Corollary of Euler's Theorem}
    \begin{block}{An interesting fact}
     No matter how we redraw a planar graph it will always have the same number of faces
    \end{block}
    
    
    \textbf{Proof}\newline
    $f=2-v+e $
    
        
    \end{frame}
\end{section}

\begin{section}{Closing Remarks}

    


\begin{frame}{Acknowledgement}
\small{
\begin{itemize}
    \item \url{https://mathweb.ucsd.edu/~gptesler/154/slides/154_planar_20-handout.pdf}
    \item \url{https://www.slideserve.com/thor/planar-graphs}
    \item \url{https://en.wikipedia.org/wiki/Planar_graph}
    \end{itemize}
}
\end{frame}



\begin{frame}{Further Reads}
\url{https://ieeexplore.ieee.org/document/5206863} \newline \newline
A very interesting application of planar graph is in ComputerVision \\
Efficient planar graph cuts with applications in Computer Vision\\
by Frank R. Schmidt et al.
    
\end{frame}

\end{section}


\end{document}
